%\documentclass{beamer} 
\documentclass[handout]{beamer} 
\usetheme{Ilmenau}
\usepackage{graphicx,verbatim,hyperref}
\usepackage{textpos}

\usecolortheme{beaver}
\useinnertheme{default}
\setbeamertemplate{itemize item}[triangle]
\setbeamertemplate{itemize subitem}[triangle]
\setbeamertemplate{itemize subsubitem}[circle]
\setbeamertemplate{enumerate items}[default]
\setbeamertemplate{blocks}[upper=block head,rounded]
\setbeamercolor{item}{fg=black}
\usefonttheme{serif} %should allow ccfonts to take effect

\usepackage{cite}
\usepackage{xcolor,bm}
\usepackage{amsbsy,amssymb, amsmath, amsthm}
\usepackage{booktabs}
%David miller's fonts
	\usepackage[T1]{fontenc}
	\usepackage[boldsans]{ccfonts}
	\usepackage[euler-hat-accent]{eulervm}

\newcommand{\al}{\alpha}
\newcommand{\expect}{\mathbb{E}}
\newcommand{\Bt}{B(\bm{\tau^a})}
\newcommand{\bta}{\bm{\tau^a}}
\newcommand{\btn}{\bm{\tau^{tw}}}
\newcommand{\ga}{\gamma}
\newcommand{\ve}{\varepsilon}
\newcommand{\ta}{\theta}

\newenvironment{changemargin}[2]{% 
  \begin{list}{}{% 
    \setlength{\topsep}{0pt}% 
    \setlength{\leftmargin}{#1}% 
    \setlength{\rightmargin}{#2}% 
    \setlength{\listparindent}{\parindent}% 
    \setlength{\itemindent}{\parindent}% 
    \setlength{\parsep}{\parskip}% 
  }% 
  \item[]}{\end{list}} 
	
	\let\Tiny=\tiny


\title[Temporary Trade Barriers: When Will They End?\hspace{2.15in}\insertframenumber/\inserttotalframenumber]{Temporary Trade Barriers: \\ When Will They End?}
\author[Kristy Buzard]{\texorpdfstring{Kristy Buzard\newline Syracuse University and The Wallis Institute  \newline\url{kbuzard@syr.edu}}{Kristy Buzard}}
\date{May 13, 2016}

\begin{document}
\maketitle
%\insertpresentationendpage removed b/c of appendix




\section{Overview}
\subsection{Preview}
\begin{frame}
\frametitle{P}
In 
\pause
\begin{itemize}[<+->]
	\item r
  \item g
	\item a
\end{itemize}
\end{frame}


\begin{frame}{Preview of Results}
\begin{itemize}[<+->]
	\item H
		\begin{itemize}
			\item R
		\end{itemize}
	\item E
		\begin{itemize}
			\item P
			\item R
			\item T
		\end{itemize}
	\item T
\end{itemize}
\end{frame}

\begin{frame}{Outline of Talk}
\begin{enumerate}[<+->]
	\item B
	\item E
	\item R
	\item F
	\item Conclusion
\end{enumerate}
\end{frame}

 
\begin{comment}
\subsection{}
\begin{frame}
\frametitle{Related Literature}
\small Protection for Sale: Grossman $\&$ Helpman (1994)
\begin{itemize}
  \item \footnotesize Empirics: Goldberg $\&$ Maggi (1999), Gawande $\&$ Bandyopadhyay (2000), Mitra, Thomakos, $\&$  Ulubasoglu (2002)
  \item \footnotesize Mitra, Thomakos, $\&$  Ulubasoglu (2006), Bombardini (2008)
	\item \footnotesize Trade Wars and Trade Talks: Grossman $\&$ Helpman (1995)
\end{itemize}

\vskip.05in
\small Political economy shocks
\begin{itemize}
	\item \footnotesize Feenstra $\&$ Lewis (1991), Bagwell $\&$ Staiger (2001, 2005)
\end{itemize}

\vskip.05in
\small Separated powers
\begin{itemize}
	\item \footnotesize Mansfield, Milner $\&$ Rosendorff (2000), Song (2008)
\end{itemize}

\vskip.05in
\small Political uncertainty
\begin{itemize}
	\item \footnotesize Milner $\&$ Rosendorff (1997), Le Breton $\&$ Zaporozhets (2007) %legislators are agents, symmetric case very hard
\end{itemize}
	
		%\item Autocracy vs. Democracy: Mitra, Thomakos and Ulubasoglu (2002), Aidt and Gassebner (2010)
\end{frame}
\end{comment}

\section{Model}
\subsection{Economic and Political Structure}
\begin{frame}{Timeline}
\pause
\begin{enumerate}
	\item {\bfseries Formation}
		\begin{enumerate}[i.]
			\pause
			\item Executives set trade policy in international agreement
		\end{enumerate}
	\pause
	\item \textbf{Ratification / Maintenance}
		\begin{enumerate}[i.]
			\pause
			\item Firms lobby legislatures to break agreement
			\pause
			\item Uncertainty is resolved
			\pause
			\item Legislatures decide to break or abide by agreement
			\pause
		\end{enumerate}
	\item \textbf{Trade War} (if agreement is broken)
	\pause
		\begin{enumerate}[i.]
			\item Firms lobby legislatures to set high trade-war tariff
			\pause
			\item Uncertainty is resolved
			\pause
			\item Legislatures decide trade-war tariff
		\end{enumerate}
\pause
	\item Private actors make production, consumption decisions
\end{enumerate}
\end{frame}


\begin{frame}{Economy}
\begin{itemize}
	\item Two countries: home and foreign (${}^*$)
	\item Separable in three goods: $X$ and $Y$ (traded) and numeraire
	\item Demand identical for both goods in both countries
	\item Supply: $Q_X^*(P_X) > Q_X(P_X)$ $\forall P_X$; symmetric for $Y$ 
		\begin{itemize}
			\item Home net importer of $X$, net exporter of $Y$
		\end{itemize}
\end{itemize}

\vskip.2in
\pause
Home levies $\tau$ on $X$, Foreign levies $\tau^*$ on $Y$
\pause
\begin{itemize}
	\item $P_X=P_X^W + \tau$ and $\pi_X(P_X)$ increasing in $\tau$
\end{itemize}

\pause
\vskip.2in
Non-tradable specific factors motivates political activity


\end{frame}


\begin{frame}{Political Structure}
In each country (focus on Home):
\pause
\begin{itemize}[<+->]
	\item A Unitary Executive
		\begin{itemize}
			\item Delegated authority to make trade agreement
		\end{itemize}
		%\pause
	\item A Non-unitary Legislature
		\begin{itemize}[<+->]
			\item Can withdraw delegation, break agreement, and set trade-war tariff
			\item Susceptible to influence of lobbying
			\item Decision determined by median legislator
		\end{itemize}
%\pause
	\item A Single Lobby
		\begin{itemize}
			\item Represents import-competing sector, $X$ ($Y$ in foreign)
		\end{itemize}	
\end{itemize}

\end{frame}


\subsection{The Players}
\begin{frame}
\frametitle{Executive Branch}
Trade agreement negotiated by unitary executive:
$$W_E = \mathit{CS}_X(\tau) + \mathit{CS}_Y(\tau^*) + \ga_E \pi_X(\tau) + \pi_Y(\tau^*) + \mathit{TR}(\tau)$$

\vskip.2in
\begin{itemize}
	\item $\mathit{CS_i(\cdot)}$: consumer surplus
	\item $\pi_X(\tau)$: profits of import-competing industry
	\item $\pi_Y(\tau^*)$: profits of exporting industry
	\item $\mathit{TR}(\tau)$: tariff revenue
	\item $\ga_E$: weight on profits in the import-competing industry
\end{itemize}


\end{frame}


\begin{frame}{Non-Unitary Legislature}
  Decisions determined by preferences of Median Legislator:
\[
  W_{\mathit{ML}} = \mathit{CS}_X(\tau) + \mathit{CS}_Y(\tau^*) + \ga(e,\ta) \pi_X(\tau) + \pi_Y(\tau^*) + \mathit{TR}(\tau)
\]
\vskip-.1in
\pause
\begin{itemize}
	\item $\ga(e,\ta)$: weight on import-competing industry profits
		\begin{itemize}
			\pause
			\item $e$: lobbying effort
			\pause
			\item $\ta$: uncertain element in determination of ML's identity
		\end{itemize}
\end{itemize}

\pause
\vskip.1in
\begin{beamerboxesrounded}[upper=palette tertiary, shadow=true]{Assumptions on $\ga(e,\ta)$}
\begin{enumerate}
  \pause
	\item $\ga(e,\ta)$ is increasing and concave in $e$ for all $\ta \in \Theta$.
  \pause
  \item $\ga(e,\ta) \geq \ga_E \geq 1 \ \forall \ta$ 
\end{enumerate}
\end{beamerboxesrounded}
\end{frame}


\begin{frame}
\frametitle{Lobby}
\vskip.2in
Lobby chooses effort to maximize:
\[
  \left\{1-\Pr\left[ \text{AD Renewal}\right]\right\} \ \pi(\tau^a) + \Pr\left[ \text{AD Renewal} \right]  \pi(\tau^{\mathit{ad}})  - e
\]

\vskip.1in
\begin{itemize}
	\item $e$: Lobbying effort
	\item $\tau^a$: home import tariff under trade agreement
	\item $\tau^{\textit{ad}}$: home import tariff equivalent under anti-dumping duties
\end{itemize}

\end{frame}


\section{Results}
\subsection{Results: The Certain Case}
\begin{frame}{Timeline}
\begin{enumerate}
	\item \textbf{Formation}
		\begin{enumerate}[i.]
			\item Executives set trade policy in international agreement
		\end{enumerate}
	\item \textbf{Ratification / Maintenance}
		\begin{enumerate}[i.]
			\item Firms lobby legislatures to break agreement
			\item Legislatures decide to break or abide by agreement
		\end{enumerate}
	\item {\color{gray} \textbf{Trade War} (if agreement is broken)}
		\begin{enumerate}[i.]
			\item {\color{gray} Firms lobby legislatures to set high trade-war tariff}
			\item {\color{gray} Legislatures decide trade-war tariff}
		\end{enumerate}
	\item {\color{gray} Private actors make production, consumption decisions}
\end{enumerate}
\end{frame}

\begin{frame}{Highlight: Separation of Powers}
\pause
\textbf{Legislature}
\pause
\begin{itemize}
	\item Breaks agreement if median legislator prefers $\tau^{\mathit{ad}}$ to $\tau^a$
\end{itemize}

\pause
\vskip.1in
\textbf{Lobby}
\pause
\begin{itemize}[<+->]
	\item Given the $(\tau^a,\tau^{*a})$ it faces, lobby knows what $e_b$ is required to break the agreement
 	\item Lobby pays this $e_b$ if: \hskip.2in $\pi(\tau^{\mathit{tw}}) - e > \pi(\tau^a)$
\end{itemize}

\pause
\vskip.13in
\textbf{Executives}
\pause
\begin{itemize}[<+->]
	\item Set $(\tau^a,\tau^{*a})$ to make paying $e_b$ unprofitable
		\begin{itemize}
			\item[$\Rightarrow$] $e_b=0$, agreement remains in force
		\end{itemize}
	\item High tariffs, no lobbying, no trade disruptions
\end{itemize}
\end{frame}


\subsection{Full Results}
\begin{frame}{Political Uncertainty Illustration}
\pause
An Example (Bagwell $\&$ Staiger 2005)
\pause
\begin{itemize}[<+->]
	\item $D(P_i) = 1 - P_i$
	\item $Q_X(P_X) = \frac{P_X}{2}$, $Q_Y(P_Y) = P_Y$
	\item $P_X^W = \frac{4-3\tau}{7}$, $P_X = \frac{4+4\tau}{7}$
	\item $\ga(e,\ta) = 1.25 + e^{0.2} + \ta$ 
		\begin{itemize}
			\item $\ta \sim U[-0.25,0.25]$
		\end{itemize}
	\item $\ga_E = 1$
\end{itemize}

\end{frame}


\begin{frame}{Timeline}
\begin{enumerate}
	\item {\color{gray} \textbf{Formation}}
		\begin{enumerate}[i.]
			\item {\color{gray} Executives set trade policy in international agreement}
		\end{enumerate}
	\item \textbf{Ratification / Maintenance}
		\begin{enumerate}[i.]
			\item {\color{gray} Firms lobby legislatures to break agreement}
			\item Legislatures decide to break or abide by agreement
		\end{enumerate}
	\item {\color{gray} \textbf{Trade War} (if agreement is broken)}
		\begin{enumerate}[i.]
			\item {\color{gray} Firms lobby legislatures to set high trade-war tariff}
			\item {\color{gray} Legislatures decide trade-war tariff}
		\end{enumerate}
	\item {\color{gray} Private actors make production, consumption decisions}
\end{enumerate}
\end{frame}


\begin{frame}{Legislature}
  Legislature breaks trade agreement if median legislator's utility is higher under trade war than trade agreement
  \begin{itemize}
		\item Median legislator's identity is uncertain through $\ta$
	\end{itemize}
	
	\pause
  \vskip.2in
  Probability that Legislature breaks agreement: \\
	\pause
  \vskip.1in
    $b(e,\tau^a,\tau^{*a},\tau^{\textit{ad}},\ta)$ probability median legislator prefers $\tau^{tw}$ to $\tau^a$ for a given $\ta$
\end{frame}




\begin{frame}{Timeline}
\begin{enumerate}
	\item {\color{gray} \textbf{Formation}}
		\begin{enumerate}[i.]
			\item {\color{gray} Executives set trade policy in international agreement}
		\end{enumerate}
	\item \textbf{Ratification / Maintenance}
		\begin{enumerate}[i.]
			\item Firms lobby legislatures to break agreement
			\item {\color{gray} Legislatures decide to break or abide by agreement}
		\end{enumerate}
	\item {\color{gray} \textbf{Trade War} (if agreement is broken)}
		\begin{enumerate}[i.]
			\item {\color{gray} Firms lobby legislatures to set high trade-war tariff}
			\item {\color{gray} Legislatures decide trade-war tariff}
		\end{enumerate}
	\item {\color{gray} Private actors make production, consumption decisions}
\end{enumerate}
\end{frame}




\section{Conclusion}
\subsection{}
\begin{frame}{Future Work}
\begin{itemize}[<+->]
	\item C
	\item E
	\item P
	\item A
\end{itemize}
\end{frame}


\begin{frame}{Conclusion}
\begin{itemize}[<+->]
	\item S
	\item E
	\item F
\end{itemize}

\end{frame}



\end{document}