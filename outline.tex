\documentclass[12pt]{article}

\addtolength{\textwidth}{1.4in}
\addtolength{\oddsidemargin}{-.7in} %left margin
\addtolength{\evensidemargin}{-.7in}
\setlength{\textheight}{8.5in}
\setlength{\topmargin}{0.0in}
\setlength{\headsep}{0.0in}
\setlength{\headheight}{0.0in}
\setlength{\footskip}{.5in}
\renewcommand{\baselinestretch}{1.0}
\setlength{\parindent}{0pt}
\linespread{1.1}

\usepackage[pdftex,
bookmarks=true,
bookmarksnumbered=false,
pdfview=fitH,
bookmarksopen=true]{hyperref}

\usepackage{amssymb, amsmath, amsthm, bm}
\usepackage{graphicx,csquotes,verbatim}
\usepackage[backend=biber,block=space,style=authoryear]{biblatex}
\setlength{\bibitemsep}{\baselineskip}
\usepackage[american]{babel}
%dell laptop
\addbibresource{C:/Users/Kristy/Dropbox/Research/xBibs/tradeagreements.bib}
%\addbibresource{C:/Users/Kristy/Documents/Dropbox/Research/xBibs/tradeagreements.bib}
\renewcommand{\newunitpunct}{,}
\renewbibmacro{in:}{}


\DeclareMathOperator*{\argmax}{arg\,max}
\usepackage{xcolor}
\hbadness=10000

\newtheorem{proposition}{Proposition}
\newcommand{\ve}{\varepsilon}
\newcommand{\ov}{\overline}
\newcommand{\un}{\underline}
\newcommand{\ta}{\theta}
\newcommand{\al}{\alpha}
\newcommand{\Ta}{\Theta}
\newcommand{\expect}{\mathbb{E}}
\newcommand{\Bt}{B(\bm{\tau^a})}
\newcommand{\bta}{\bm{\tau^a}}
\newcommand{\btn}{\bm{\tau^n}}
\newcommand{\btw}{\bm{\tau^{tw}}}
\newcommand{\ga}{\gamma}
\newcommand{\Ga}{\Gamma}
\newcommand{\de}{\delta}

\begin{document}
\begin{center}
  Preliminary Outline for Dispute Length Project
\end{center}

\vskip.3in
\section{Motivation}
Need to motivate question
\begin{itemize}
	\item How long do deviations from trade agreement tariffs last? What are the determinants of these deviations?
	\item If lobbies have to exert effort to achieve higher-than-MFN tariffs, when will it be worthwhile for them to do so?
	\item Whether it's a dispute or it's a temporary trade barrier (TTB) like AD or escape clause that has not been disputed, it won't be granted for no reason
\end{itemize}

\vskip.3in
\section{Main Idea}
Adapt SOP model to predict whether anti-dumping measures get renewed
\begin{itemize}
	\item Note that this is not trade war: foreign is applying $\tau^{*a}$ in most / all cases
		\begin{itemize}
			\item \textbf{Q}: Are all cases of renewal ones of no punishment, i.e. target country is applying MFN tariff?
		\end{itemize}
	\item When is it worth it for lobby to exert effort to renew AD measure?
	\item Lobby must be able to trigger the AD measure in the first place
		\begin{itemize}
			\item This means disputes/non-adherence to MFN tariffs must happen on eqm path
			\item In my model, it is symmetric political uncertainty about how ITC will rule. Why would there be uncertainty?
				\begin{itemize}
					\item Directly about strength of evidence? (indirectly about retaliation / dispute?)
					\item Differential valuation about harm to industry---how central the industry is, how politically powerful
					\item \textbf{Q}: What are all the factors that have weight in ITCs decision-making? Are they influenced by other political factors? 
					\textbf{A}: (Per Chad's email on 4/26/16, his and Gene Grossman's 2008 WTR) The data is useless b/c Commerce has never used counterfactual analysis.
					\item Does Congressional uncertainty transfer over to ITC uncertainty? NEED to argue that it does if we're going to use the political uncertainty data that I'm generating
				\end{itemize}
		\end{itemize}
	\item In this setup, need ``dispute'' to last for 5 periods (years); 4 for safeguard
		\begin{itemize}
			\item Then can extend it.
			\item \textbf{Q}: for five more years? \textbf{A}: broadly looks true in data (to my casual glance)
			\item Safeguards are rarely renewed (retaliation allowed after 3 years)
		\end{itemize}
	\item Why would there be variation in one lobby's incentives between $t=1$ (original application of AD) and $t=6$ when it comes up for renewal?
		\begin{itemize}
			\item Political uncertainty could be an answer, \textit{and} it varies across industry
			\item \textbf{Q}: Is this a plausible story?
		\end{itemize}
	\item Also have to adapt model to cross-industry to get necessary variation
		\begin{itemize}
			\item I've already done some of this leg work for the NSF proposals, thinking about PTA project
		\end{itemize}
\end{itemize}

\vskip.5in
\section{Median Legislator's Condition}
\begin{itemize}
	\item I believe I have to change the legislature's condition to be more like the cheater's payoff for this context
		\[
		  W\left( \tau^{AD},\tau^{*a},\ga(e,\ta)\right) > W\left( \bta,\ga(e,\ta)\right)
		\]
		\begin{itemize}
			\item Need to make sure this is not always the case.
				\begin{itemize}
					\item Median legislator still has to balance (weighted) producers and consumers.
					\item If $\ga = 1$, would pick optimal tariff.
					\item If $\ga$ is so low that $\tau^N < \tau^a$, then agreement will hold. If $\tau^a < \tau^N < \tau^{AD}$, depends on which is closer in welfare terms 
				\end{itemize}
			\item Seems to work okay in Matlab example: just pushes up break probability, trade agreement tariff; reduces gamma and effort (``SOP$\_$example.m'')
			\item Exec's SOC doesn't matter: going to take $\bta$ as given. But need to worry about lobby and leg SOCs. I've convinced myself that the relevant results from my JMP go through for this objective function, so SOCs go through too (given the right assumptions to get concavity for leg, or possibly the argumentation from trade war section supplied by RoIE editor)
		\end{itemize}
	\item There could also be uncertainty about the probability that foreign will dispute the AD measure; that could change from the original to the renewal
		\begin{itemize}
			\item Could reduce form this, or not
		\end{itemize}
\end{itemize}


\vskip.5in
\section{Cross-industry Variation}
\begin{itemize}
	\item Lobby facing same uncertainty, behaving in same manner may get different outcome in the two draws (five years apart)
		\begin{itemize}
			\item In first round, $\tau^{AD}$ is endogenous. It's exogenous in second round of play.
			\item \textbf{Q}: Is it really exogenous? What does data say about whether renewals are at same level as original TTB?
			\item \textbf{A}: The statements are of the form ``As a result of the respective determinations by the Department and the ITC that revocation of the antidumping duty order on CTL plate from the PRC and termination of the Agreements on CTL plate from Russia and Ukraine would likely lead to continuation or recurrence of dumping and material injury to an industry in the United States, pursuant to section 751(d)(2) of the Act, the Department hereby gives notice of the continuation of the antidumping duty order on CTL plate from the PRC and the continuation of the Agreements on CTL plate from Russia and Ukraine. The effective dates of continuation will be the date of publication in the Federal Register of this Continuation Notice. Pursuant to sections 751(c)(2) and 751(c)(6) of the Act, the Department intends to initiate the next five-year sunset reviews of the antidumping duty order on CTL plate from the PRC and the Agreements on CTL plate from Russia and Ukraine not later than November 2019.'' \url{https://www.federalregister.gov/articles/2015/12/21/2015-32022/continuation-of-antidumping-duty-order-on-certain-cut-to-length-carbon-steel-plate-from-the-peoples}
		\end{itemize}
	\item Industry / lobby gets richer / more insulated for five years (or poorer if not insulated enough)
		\begin{itemize}
			\item This could lead to differences in budget constraint if that were in model
			\item May not need budget constraint if extra budget allows them to invest in technology or politics
				\begin{itemize}
					\item Come to question of whether protection and technological upgrading are complements or substitutes
					\item Lobbies that have more to gain have more opportunity to \textit{either} gather strength to become more competitive \textit{or} become more politically powerful to seek more protection
						\begin{itemize}
							\item Is there an incentive to \textit{not} get stronger technologically?
						\end{itemize}
					\item Perhaps some cross-industry measure of restraints on political strategy that would push toward substituting to technological
				\end{itemize}
			\item This could lead to differences in ability to deal with technological gap with foreign competitors 
				\begin{itemize}
					\item \textbf{Q}: This is one of the arguments for escape clause, no?
				\end{itemize}
		\end{itemize}
	\item Even if AD economic conditions can't be measured / don't bind, doesn't mean that real economic conditions don't play into ITC/DOC's decision-making process
		\begin{itemize}
			\item DOC initiates sunset review, determines whether dumping would continue
			\item ITC investigates whether expiry of AD would lead to recurrence or continuation of injury
		\end{itemize}
	\item Uncertainty could change, so behavior would change (this would be hard to pick up in the data that I have)
		\begin{itemize}
			\item If the dist'n doesn't change, only the outcome, I'd just be observing draws from the same distribution, driven by same behavior.
			\item I guess this is something I want to be able to tease apart? \textit{If} uncertainty plays an important role, is it just the outcome that's changing, or the underlying uncertainty and thus the behavior?
			\item Can I get at this with the data I have (i.e. \textit{will} have, shortly)?
		\end{itemize}
\end{itemize}


\vskip1in
\section{Model}
What are the \textit{essential} insights/predictions I want to capture?
\begin{enumerate}
	\item uncertainty about ITC's preferences impacts lobby's incentives to exert effort
	\item cross-industry variation in whether AD duties are renewed
\end{enumerate}

\vskip.2in
Do I need two stages? What happens with only one stage?
\begin{enumerate}
	\item Given $\tau^{AD}$, lobby chooses $e$ given the dist'n of $\ta$
	\item $\ta$ realized
	\item ITC decides between $\tau^{AD}$ and $\tau^a$ using
	  \[
		  W\left(\ga(e,\ta), \tau^{AD},\tau^{*a}\right) > W\left(\ga(e,\ta), \tau^a, \tau^{*a}\right)
		\]
\end{enumerate}

\vskip.2in
This is simple because there's only one lobby, action in only one country
\begin{itemize}
	\item Although there may be pressure from foreign
	\item Or in a three-country model: third country benefits from discrimination against foreign $\left( * \right)$
\end{itemize}

\vskip.5in
Notes
\begin{enumerate}
	\item I don't think a more general model like BS1999 will work; I need to be able to pick apart the elements of the welfare function to separate out $\ga(e,\ta)$ in order to do my proofs
\end{enumerate}

\vskip1in
Next steps:
\begin{enumerate}
	\item Email Chad with questions
	\item How do I make this vary across industry? $\pi$ function, $\ga$, $\ta$, $\tau^a$, $\tau^{AD}$
\end{enumerate}

\newpage
\section{Chad and Maurizio's project}
\begin{itemize}
	\item Chad and Maurizio Zanardi are working on a paper on AD 5-year reviews
		\begin{itemize}
			\item After five years, they come up for review 
				\begin{itemize}
					\item Some AD measures get removed, some not, some go to dispute
					\item This is, of course, conditional on getting to five years
				\end{itemize}
			\item They have the data, but are not exploiting cross-industry variation
				\begin{itemize}
					\item Instead, aggregate variation, things like recessions, exchange rates
				\end{itemize}
			\item They don't have a theory for the cross-industry variation, because the economic determinants are meaningless after five years
				\begin{itemize}
					\item No injury, import surges: they've been protected for five years. No variation in new economic date b/c they've been insulated
					\item What's the economic test? There really isn't one. ``Would there be injury if we removed the duty?''
					\item Politics could be that theory (my theory from above)
						\begin{itemize}
							\item Q: Does hiring of lawyers for AD procedure get caught up in LDA data?
						\end{itemize}
				\end{itemize}
		\end{itemize}
\end{itemize}

		
\end{document}