\documentclass[12pt]{article}

\addtolength{\textwidth}{1.4in}
\addtolength{\oddsidemargin}{-.7in} %left margin
\addtolength{\evensidemargin}{-.7in}
\setlength{\textheight}{8.5in}
\setlength{\topmargin}{0.0in}
\setlength{\headsep}{0.0in}
\setlength{\headheight}{0.0in}
\setlength{\footskip}{.5in}
\renewcommand{\baselinestretch}{1.0}
\setlength{\parindent}{0pt}
\linespread{1.1}

\usepackage{amssymb, amsmath, amsthm, bm}
\usepackage{graphicx,csquotes,verbatim}
\usepackage[backend=biber,block=space,style=authoryear]{biblatex}
\setlength{\bibitemsep}{\baselineskip}
\usepackage[american]{babel}
%dell laptop
\addbibresource{C:/Users/Kristy/Dropbox/Research/xBibs/tradeagreements.bib}
%\addbibresource{C:/Users/Kristy/Documents/Dropbox/Research/xBibs/tradeagreements.bib}
\renewcommand{\newunitpunct}{,}
\renewbibmacro{in:}{}


\DeclareMathOperator*{\argmax}{arg\,max}
\usepackage{xcolor}
\hbadness=10000

\newtheorem{proposition}{Proposition}
\newcommand{\ve}{\varepsilon}
\newcommand{\ov}{\overline}
\newcommand{\un}{\underline}
\newcommand{\ta}{\theta}
\newcommand{\al}{\alpha}
\newcommand{\Ta}{\Theta}
\newcommand{\expect}{\mathbb{E}}
\newcommand{\Bt}{B(\bm{\tau^a})}
\newcommand{\bta}{\bm{\tau^a}}
\newcommand{\btn}{\bm{\tau^n}}
\newcommand{\btw}{\bm{\tau^{tw}}}
\newcommand{\ga}{\gamma}
\newcommand{\Ga}{\Gamma}
\newcommand{\de}{\delta}

\begin{document}
\begin{center}
  Preliminary Outline for Dispute Length Project
\end{center}

\vskip.3in
Need to motivate question
\begin{itemize}
	\item How long do deviations from trade agreement tariffs last? What are the determinants of these deviations?
	\item If lobbies have to exert effort to achieve higher-than-MFN tariffs, when will it be worthwhile for them to do so?
	\item Whether it's a dispute or it's a measure (what does Chad call these?) like AD or escape clause that has not been disputed, it won't be granted for no reason
\end{itemize}

\vskip.3in
Main idea: adapt SOP model to predict whether anti-dumping measures get renewed
\begin{itemize}
	\item Note that this is not trade war: foreign is applying $\tau^{*a}$ in most / all cases
		\begin{itemize}
			\item \textbf{Q}: Are all cases of renewal ones of no punishment, i.e. target country is applying MFN tariff?
		\end{itemize}
	\item When is it worth it for lobby to exert effort to renew AD measure?
	\item Lobby must be able to trigger the AD measure in the first place
		\begin{itemize}
			\item This means disputes/non-adherence to MFN tariffs must happen on the equilibrium path
			\item Need uncertainty, asymmetric information, something
			\item In my model, it is symmetric political uncertainty about how ITC will rule. Why would there be uncertainty?
				\begin{itemize}
					\item Directly about strength of evidence? (indirectly about retaliation / dispute)
					\item Differential valuation about harm to industry---how central the industry is, how politically powerful
					\item \textbf{Q}: What are all the factors that have weight in ITCs decision-making? Are they influenced by other political factors?
					\item Does Congressional uncertainty transfer over to ITC uncertainty?
				\end{itemize}
		\end{itemize}
	\item In this setup, need ``dispute'' to last for 5 periods (years)
		\begin{itemize}
			\item Then can extend it.
			\item \textbf{Q}: for five more years?
		\end{itemize}
	\item Why would there be variation in one lobby's incentives between $t=1$ (original application of AD) and $t=6$ when it comes up for renewal?
		\begin{itemize}
			\item Uncertainty could be an answer, \textit{and} it varies across industry
			\item \textbf{Q}: Is this a plausible story?
		\end{itemize}
	\item Also have to adapt model to cross-industry to get necessary variation
		\begin{itemize}
			\item I've already done some of this leg work for the NSF proposals, thinking about PTA project
		\end{itemize}
\end{itemize}

\vskip.5in
Median Legislator's Condition
\begin{itemize}
	\item I believe I have to change the legislature's condition to be more like the cheater's payoff for this context
		\[
		  W_{ML}\left( \tau^{AD},\tau^{*a},\ga(e,\ta)\right) > W_{ML}\left( \bta,\ga(e,\ta)\right)
		\]
		\begin{itemize}
			\item Need to make sure this is not always the case.
				\begin{itemize}
					\item Median legislator still has to balance (weighted) producers and consumers.
					\item If $\ga = 1$, would pick optimal tariff.
					\item If $\ga$ is so low that $\tau^N < \tau^a$, then agreement will hold. If $\tau^a < \tau^{AD} < \tau^{AD}$, depends on which is closer in welfare terms 
				\end{itemize}
			\item Seems to work okay in Matlab example: just pushes up break probability, trade agreement tariff; reduces gamma and effort (``SOP$\_$example.m'')
			\item Need to check exec's SOC
		\end{itemize}
	\item There could also be uncertainty about the probability that foreign will dispute the AD measure; that could change from the original to the renewal
\end{itemize}


\vskip.5in
Possible cross-industry variation
\begin{itemize}
	\item Lobby facing same uncertainty, behaving in same manner may get different outcome in the two draws (five years apart)
		\begin{itemize}
			\item In first round, $\tau^{AD}$ is endogenous. It's exogenous in second round of play.
		\end{itemize}
	\item Industry / lobby gets richer / more insulated for five years
		\begin{itemize}
			\item This could lead to differences in budget constraint if that were in model
			\item May not need budget constraint if extra budget allows them to invest in technology
				\begin{itemize}
					\item Come to question of whether protection and technological upgrading are complements or substitutes
					\item Lobbies that have more to gain have more opportunity to \textit{either} gather strength to become more competitive \textit{or} become more politically powerful to seek more protection
					\item Perhaps some cross-industry measure of restraints on political strategy that would push toward substituting to technological
				\end{itemize}
			\item This could lead to differences in ability to deal with technological gap with foreign competitors 
				\begin{itemize}
					\item \textbf{Q}: This is one of the arguments for escape clause, no?
				\end{itemize}
		\end{itemize}
	\item Even if AD economic conditions can't be measured / don't bind, doesn't mean that real economic conditions don't play into ITC's decision-making process
	\item Uncertainty could change, so behavior would change (this would be hard to pick up in the data that I have)
\end{itemize}

\newpage
\begin{itemize}
	\item Chad and Maurizio Zanardi are working on a paper on AD 5-year reviews
		\begin{itemize}
			\item After five years, they come up for review 
				\begin{itemize}
					\item Some AD measures get removed, some not, some go to dispute
					\item This is, of course, conditional on getting to five years
				\end{itemize}
			\item They have the data, but are not exploiting cross-industry variation
				\begin{itemize}
					\item Instead, aggregate variation, things like recessions, exchange rates
				\end{itemize}
			\item They don't have a theory for the cross-industry variation, because the economic determinants are meaningless after five years
				\begin{itemize}
					\item No injury, import surges: they've been protected for five years. No variation in new economic date b/c they've been insulated
					\item What's the economic test? There really isn't one. ``Would there be injury if we removed the duty?''
					\item Politics could be that theory (my theory from above)
						\begin{itemize}
							\item Q: Does hiring of lawyers for AD procedure get caught up in LDA data?
						\end{itemize}
				\end{itemize}
		\end{itemize}
\end{itemize}

		
\end{document}