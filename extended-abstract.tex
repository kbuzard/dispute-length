\documentclass[12pt]{article}

\addtolength{\textwidth}{1.4in}
\addtolength{\oddsidemargin}{-.7in} %left margin
\addtolength{\evensidemargin}{-.7in}
\setlength{\textheight}{8.5in}
\setlength{\topmargin}{0.0in}
\setlength{\headsep}{0.0in}
\setlength{\headheight}{0.0in}
\setlength{\footskip}{.5in}
\renewcommand{\baselinestretch}{1.0}
\setlength{\parindent}{0pt}
\linespread{1.2}

\usepackage{amssymb, amsmath, amsthm, bm}
\usepackage{graphicx,csquotes,verbatim}
\usepackage[backend=biber,block=space,style=authoryear]{biblatex}
\setlength{\bibitemsep}{\baselineskip}
\usepackage[american]{babel}
%dell laptop
\addbibresource{C:/Users/Kristy/Dropbox/Research/xBibs/tradeagreements.bib}
%\addbibresource{C:/Users/Kristy/Documents/Dropbox/Research/xBibs/tradeagreements.bib}
\renewcommand{\newunitpunct}{,}
\renewbibmacro{in:}{}


\DeclareMathOperator*{\argmax}{arg\,max}
\usepackage{xcolor}
\hbadness=10000

\newtheorem{proposition}{Proposition}
\newcommand{\ve}{\varepsilon}
\newcommand{\ov}{\overline}
\newcommand{\un}{\underline}
\newcommand{\ta}{\theta}
\newcommand{\al}{\alpha}
\newcommand{\Ta}{\Theta}
\newcommand{\expect}{\mathbb{E}}
\newcommand{\Bt}{B(\bm{\tau^a})}
\newcommand{\bta}{\bm{\tau^a}}
\newcommand{\btn}{\bm{\tau^n}}
\newcommand{\btw}{\bm{\tau^{tw}}}
\newcommand{\ga}{\gamma}
\newcommand{\Ga}{\Gamma}
\newcommand{\de}{\delta}

\begin{document}
\begin{center}
  Temporary Trade Barriers: When Will They End?\\
	Kristy Buzard
\end{center}

\vskip.2in
Today's world trading system is largely governed by a system in which countries commit to tariff bindings and yet raise their tariffs above those bindings through a variety of temporary trade barriers (TTBs) on a not-infrequent basis (Bown 2011\footnote{Bown, Chad P., ``Taking Stock of Antidumping, Safeguards and Countervailing Duties, 1990-2009,'' The World Economy v34, n12 (December 2011): 1955-1998.}). A significant literature has explored the question of which industries receive protection through TTBs and under what conditions this happens. This project asks a related but distinct question: given that a product receives protection, what determines the duration of the protection it receives? \\

This question can be addressed in the context of renewals of trade remedies under the WTO agreements. Normally, temporary measures such as anti-dumping and counter-vailing duties are imposed with a five-year sunset provision that is subject to renewal by the U.S. International Trade Commission. Conditional on a TTB being granted and applied for five years, one would like to know what determines whether a renewal order is granted so that the measure continues in force. \\

In many countries, the crucial actors in the imposition of TTBs are advocates for the industry that would be protected and an administrative body that has been granted authority to make decisions about whether a given TTB is WTO-legal. In the United States, this body is the International Trade Commission (ITC). For ease of exposition and because I would ultimately like to test this theory using U.S. data, I will refer to the actors in this model as industries and the ITC. \\

In this model, there are two phases of interaction between any given industry seeking protection and the ITC. In both cases, I assume that all actors take most-favored-nation (MFN) tariffs as given. \\

In the first phase, when no TTB is in place, an industry exerts effort to convince the ITC to enact a TTB for its product. The ITC then must decide whether to maintain the MFN tariff or impose the temporary trade barrier. Importantly, we assume that the decisions of the ITC are not deterministic, with the amount of uncertainty varying by industry. \\

Given that the industry receives temporary protection in the first phase of interaction, there will be another phase of interaction when the trade barrier expires, here modeled as five years later to match the typical five years sunset provision under the WTO. At this point, the interaction is similar to that during the first phase, with the industry first exerting effort and then the ITC deciding whether to extend the TTB or to revert to the MFN tariff. The essential strategic difference is that at this point the level of the TTB is taken as exogenously set at the level determined in the first round of interaction. \\

An industry's decision about how much, if any, effort to exert depends on a number of factors. Among them are the gap between the applied tariff it faces and the protection it would receive under a TTB, the cost of seeking the TTB, and, crucially, the probability that its request will be granted. That is, the industry's incentives to seek protection, and the intensity with which it does so, depend on how much uncertainty it faces in the ITC's decision-making process. \\
 
Thus we turn to the question of how the ITC decides whether to grant a TTB. The imposition of a TTB, or continuation of one already in place, is not in general costless. We might assume then that the ITC only grants renewals when it finds the benefits outweigh the costs. If the ITC's rulings are uncertain from the point of view of industry, it must be that the industry cannot perfectly predict the outcome of any given proceeding. That is, the industry does not know with precision how the ITC weighs the costs and benefits of granting protection.\\

There are several possible sources of this uncertainty. The industrial group may not be able to predict directly the quality of evidence the ITC requires in order to be willing to provide protection through a TTB. Indirectly, this may reflect the ITC's preferences about the costs of disputes and retaliation by trade partners. Industry may also not be fully informed about how the ITC weighs the potential harms to the industry versus those to up- and down-stream industries, consumers and trading relations. This may be derived from factors such as the political influence of the industry, how central it is to the economy, and whether there is active lobbying by producers downstream of the product in question. \\

An important question in this model is why an industry that receives protection as a result of the original interaction concerning the imposition of the TTB would fail to get it renewed. At the most basic level, if we assume the industry faces the same uncertainty and responds in the same way, it could get a different draw and thus a different outcome. Moreover, having the level of the TTB set by the original interaction serves to reduce the industry's incentive to exert effort on the margin. \\

In addition, some factors that affect both the ITC's and the industry's decision-making can change significantly over the course of a five-year period. Most of these factors display at least some variation across industries. \\

Because a TTB insulates an industry from competition, when receiving such protection, an industry should be able to increase profits. The amount that profits increase will vary by industry, and the decisions about how to use the extra profits will also plausibly vary. Some industries may use this opportunity to become more competitive and thus less interested in seeking protection while others may use the added profits to become more politically powerful and thus better able to gain future protection. These decisions, and the basic question of whether protection and technological upgrading are complements or substitutes, feeds back into the decision-making process of the ITC. \\

Another interesting dimension along which there is uncertainty for the ITC is in its evaluation of the probability with which trading partners will dispute or retaliate against a TTB. This not only varies across TTBs, but may vary across time. In particular, the ITC can observe whether a dispute has been filed during the first five years of a TTB and update its beliefs about future retaliation in a way that varies across industries. \\

There is much to be learned about the workings of the world trading system by examining the duration of deviations from base tariff commitments. I plan a project whose first stage is theoretical with a second, later stage in which the theory is tested on a data set of renewal decisions for anti-dumping measures.
		
\end{document}