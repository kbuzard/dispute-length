\documentclass[12pt]{article}

\addtolength{\textwidth}{1.4in}
\addtolength{\oddsidemargin}{-.7in} %left margin
\addtolength{\evensidemargin}{-.7in}
\setlength{\textheight}{8.5in}
\setlength{\topmargin}{0.0in}
\setlength{\headsep}{0.0in}
\setlength{\headheight}{0.0in}
\setlength{\footskip}{.5in}
\renewcommand{\baselinestretch}{1.0}
\setlength{\parindent}{0pt}
\linespread{1.2}

\usepackage{amssymb, amsmath, amsthm, bm}
\usepackage{graphicx,csquotes,verbatim}
\usepackage[backend=biber,block=space,style=authoryear]{biblatex}
\setlength{\bibitemsep}{\baselineskip}
\usepackage[american]{babel}
%dell laptop
\addbibresource{C:/Users/Kristy/Dropbox/Research/xBibs/tradeagreements.bib}
%\addbibresource{C:/Users/Kristy/Documents/Dropbox/Research/xBibs/tradeagreements.bib}
\renewcommand{\newunitpunct}{,}
\renewbibmacro{in:}{}


\DeclareMathOperator*{\argmax}{arg\,max}
\usepackage{xcolor}
\hbadness=10000

\newtheorem{proposition}{Proposition}
\newcommand{\ve}{\varepsilon}
\newcommand{\ov}{\overline}
\newcommand{\un}{\underline}
\newcommand{\ta}{\theta}
\newcommand{\al}{\alpha}
\newcommand{\Ta}{\Theta}
\newcommand{\expect}{\mathbb{E}}
\newcommand{\Bt}{B(\bm{\tau^a})}
\newcommand{\bta}{\bm{\tau^a}}
\newcommand{\btn}{\bm{\tau^n}}
\newcommand{\btw}{\bm{\tau^{tw}}}
\newcommand{\ga}{\gamma}
\newcommand{\Ga}{\Gamma}
\newcommand{\de}{\delta}

\begin{document}
\begin{center}
  Temporary Trade Barriers: When Will They End?\\
	Kristy Buzard
\end{center}

\vskip.2in
Today's world trading system is largely governed by a system in which countries commit to tariff bindings and yet raise their tariffs above those bindings through a variety of temporary trade barriers (TTBs) on a not-infrequent basis (Bown 2011). A significant literature has explored the question of which industries receive protection through TTBs and under what conditions this happens. This project asks a related but distinct question: given that a product receives protection, what determines the duration of the protection it receives? \\

This question can be addressed in the context of renewals of trade remedies under the WTO agreements. Normally, temporary measures such as anti-dumping and counter-vailing duties are imposed with a five-year sunset provision that is subject to renewal by the U.S. International Trade Commission. Conditional on a TTB being granted and applied for five years, one would like to know what determines whether a renewal order is granted so that it continues in force. \\

In many countries, the crucial actors in the imposition of TTBs are advocates for the industry that would be protected and an administrative body that has been granted authority to make decisions about whether a given TTB is WTO-legal. In the United States, this body is the International Trade Commission (ITC). For ease of exposition and because I would ultimately like to test this theory using U.S. data, I will refer to the actors in this model as lobbies and the ITC. \\

In this model, there are two phases of interaction between any given industry seeking protection and the ITC. In both cases, all the actors take most-favored-nation (MFN) tariffs as given. \\

In the first phase, when no TTB is in place, an industry exerts effort to convince the ITC to enact a TTB for its product. The ITC then must decide whether the maintain the MFN tariff or impose the TTB barrier. Importantly, we assume that the decisions of the ITC are not deterministic, with the amount of uncertainty varying by industry. \\

Given that the industry receives temporary protection in the first phase of interaction, there will be another phase of interaction when the trade barrier expires, here modeled as five years later to match the typical five years sunset provision under the WTO. At this point, the interaction is essentially the same with the industry first exerting effort and then the ITC deciding whether to extend the TTB or to revert to the MFN tariff. The essential strategic difference is that at this point the level of the TTB is taken as exogenously set at the level determined in the first round of interaction. \\

An industry's decision about how much, if any, effort to exert depends on a number of factors. Among them are the gap between the applied tariff it faces and the protection it would receive under a TTB, the cost of seeking the TTB, and, crucially, the probability that its request will be granted. That it, the industry's incentives to seek protection, and the intensity with which it does so, depend on how much uncertainty it faces in the ITC's decision-making process. \\
 
Thus we turn to the question of how the ITC decides whether to grant a TTB. The imposition of a TTB, or continuation of one already in place, is not in general costless. We might assume then, that the ITC only grants renewals when it finds the benefits outweigh the costs. If the ITC's rulings are uncertain from the point of view of industry, it must be that the industry cannot perfectly predict the outcome of any given proceeding. \\

This uncertainty may derive from several sources. The industrial group may not be able to predict directly the quality of evidence the ITC requires in order to be willing to provide protection through a TTB. Indirectly, this may indirectly reflect the ITC's preferences about the costs of disputes and retaliation by trade partners. Industry may also not be fully informed about how the ITC weighs the potential harms to the industry versus those to consumers and trading relations. This may be derived from factors such as the political influence of the industry, how central it is to the economy, and whether there is active lobbying by producers downstream of the product in question.




\vskip.5in
Possible cross-industry variation
\begin{itemize}
	\item Lobby facing same uncertainty, behaving in same manner may get different outcome in the two draws (five years apart); or the incentives could be different
		\begin{itemize}
			\item In first round, $\tau^{AD}$ is endogenous. It's exogenous in second round of play.
			\item Uncertainty could be an answer, \textit{and} it varies across industry
		\end{itemize}
	\item Industry / lobby gets richer / more insulated for five years
		\begin{itemize}
			\item This could lead to differences in budget constraint if that were in model
			\item May not need budget constraint if extra budget allows them to invest in technology
				\begin{itemize}
					\item Come to question of whether protection and technological upgrading are complements or substitutes
					\item Lobbies that have more to gain have more opportunity to \textit{either} gather strength to become more competitive \textit{or} become more politically powerful to seek more protection
					\item Perhaps some cross-industry measure of restraints on political strategy that would push toward substituting to technological
				\end{itemize}
			\item This could lead to differences in ability to deal with technological gap with foreign competitors 
				\begin{itemize}
					\item \textbf{Q}: This is one of the arguments for escape clause, no?
				\end{itemize}
		\end{itemize}
	\item Even if AD economic conditions can't be measured / don't bind, doesn't mean that real economic conditions don't play into ITC's decision-making process
	\item Uncertainty could change, so behavior would change (this would be hard to pick up in the data that I have)
	\item There could also be uncertainty about the probability that foreign will dispute the AD measure; that could change from the original to the renewal
\end{itemize}

\vskip.3in
Chad and Maurizio Zanardi are working on a paper on AD 5-year reviews
\begin{itemize}
	\item They have the data, but are not exploiting cross-industry variation
		\begin{itemize}
			\item Instead, aggregate variation, things like recessions, exchange rates
		\end{itemize}
	\item They don't have a theory for the cross-industry variation, because the economic determinants are meaningless after five years
	\item Politics could be that theory (my theory from above)		
\end{itemize}

		
\end{document}